%Irfansha Shaik, 27.04.2020, Aarhus.

\documentclass[11pt]{report}

\usepackage[utf8]{inputenc}
\usepackage{geometry}
\geometry{a4paper}

\setcounter{secnumdepth}{4}

\usepackage{graphicx}

%%% PACKAGES
\usepackage{booktabs}
\usepackage{array}
\usepackage{paralist}
\usepackage{verbatim}
\usepackage{subfig}
\usepackage{hyperref}
\usepackage{natbib}
\usepackage{amsmath}
\usepackage{textcomp}

%%% HEADERS & FOOTERS
\usepackage{fancyhdr}
\pagestyle{fancy}
\renewcommand{\headrulewidth}{0pt}
\lhead{}\chead{}\rhead{}
\lfoot{}\cfoot{\thepage}\rfoot{}

%%% SECTION TITLE APPEARANCE
\usepackage{sectsty}
\allsectionsfont{\sffamily\mdseries\upshape}

%%% ToC (table of contents) APPEARANCE
\usepackage[nottoc,notlof,notlot]{tocbibind} % Put the bibliography in the ToC
\usepackage[titles,subfigure]{tocloft}
\renewcommand{\cftsecfont}{\rmfamily\mdseries\upshape}
\renewcommand{\cftsecpagefont}{\rmfamily\mdseries\upshape} % No bold!

%\DeclareMathOperator{\PSPACE}{PSPACE} % PSpace classes

\title{Symbolic Side-attack scalability in Klee}
\author{Emilie Lykke Thonsgaard \and Irfansha Shaik}
%\date{}

\begin{document}
\maketitle

\section{Planning}
\label{sec:planning}

Todos for IS:
\begin{enumerate}
\item Test the if a secret value is reachable and take advantage of heuristics.
  \begin{enumerate}
  \item use option "-exit-on-error" to stop when a solution is found.
  \end{enumerate}
\item try to initialise a new symbolic variable when needed (not at once in starting).
\end{enumerate}

\chapter{Introduction}
\label{cha:introduction}

XXX

Summary of Symbolic analysis of probabilistic programs paper:
\begin{enumerate}
\item The main challenge here is to tackle the side channel attacks.
\item Symbolic variables are added to probabilistic outcome and leaking is measured by quantification.
\item Tools to use, Symbolic PathFinder and used them on Java program
\item The key idea is to quantify the information leakage in standard functions such as Random functions.
\item While injected randomness can be used as a counter measure, it is essential to measure the effectiveness the measures.
\end{enumerate}

\section{Side channel attacks}
\label{sec:sidechannelattacks}

XXX

\chapter{Experimentation and design}
\label{cha:experimentationanddesign}

We consider a simple side channel attack via symbolic execution.
Consider the code snippet provided below:
\begin{verbatim}
int vulnerable_code(int public_input) {

if (secret >= public_input) { return 1; } # Cost is 1
else { return 2; } # Cost is 2

}
\end{verbatim}

As in any side channel attack, we need difference in some measurement (for example time) here we simply provide the cost as integer
for comparision between different paths.

Now the idea is to apply $k$-step attack, such that we extract some information on secret key.
For example, let $\{1 \dots 6\}$ be the domain of the secret (which we assume to know) and suppose we apply a $2$-step attack.
Here, we choose two public inputs such that we slice the domain such that we retrive the secret key depending on the cost measurement.
Given enough $k$, it is possible to retreive the secret key completely however it is not practical as search tree size grows exponentially.

XXX

Instructions to use for compilation, run and looking at testcases:
\begin{verbatim}
clang -I ../../include -emit-llvm -c -g ./examples/project_proto/symbolic_attack.c
klee -emit-all-errors symbolic_attack.bc 1 6 2
ktest-tool test000018.ktest
\end{verbatim}

\bibliographystyle{plain}
\bibliography{references}

\end{document}

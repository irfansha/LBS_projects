%\documentclass{report}
\documentclass[11pt,a4paper,notitlepage]{article}
\usepackage[utf8]{inputenc}
\usepackage{titling}
\usepackage{hyperref}

\title{Symbolic execution of adaptive side-channel attacks}
\author{Emilie L. Thonsgaard \and Irfansha Shaik}
\date{May 2020}

\usepackage{natbib}
\usepackage{graphicx}

\begin{document}
\begin{titlingpage}
    \maketitle
    \begin{abstract}
        In this report we review the subject of side-channel attacks and how symbolic execution is used in this context[...]
    \end{abstract}
\end{titlingpage}

\tableofcontents
\newpage
\setcounter{section}{-1}


\section{Todos for IS}
\label{sec:todosforis}

\begin{enumerate}
  \item Add first set of data to the repository.
  \item Analyse the data for any differences based on heuristics.
  \item Run the script for larger domain size and $k$-step attack.
\end{enumerate}

\newpage

\section{Introduction}
\label{cha:introduction}

[...]

\newpage

\section{Literature review}
\label{cha:literaturereview}

XXX

\subsection{Side channel attacks}
\label{sec:sidechannelattacks}

XXX 

\subsection{Symbolic execution}
\label{sec:symbolicexecution}

XXX

\newpage

\section{Research questions}
\label{sec:researchquestions}

XXX

\newpage

\section{Experimentation setup and Analysis}
\label{sec:experimentationanddesign}

To invesigate the scalability of adaptive side channel attacks in \texttt{Klee}, we first model a side channel attack and investigate the adaptive $k$-step attack in \texttt{Klee}.

\subsection{Modelling the adaptive symbolic side channel attack}
\label{subsec:modellingsscattack}

We consider a simple side channel attack via symbolic execution.
Consider the code snippet provided below:
\begin{verbatim}
int vulnerable_code(int public_input) {

if (secret >= public_input) { return 1; } // Cost is 1
else { return 2; } // Cost is 2

}
\end{verbatim}

As in any side channel attack, we need difference in some measurement (for example time) here we simply provide the cost as integer
for comparision between different paths.

Now the idea is to apply $k$-step attack, such that we extract some information on secret key.
For example, let $\{1 \dots 6\}$ be the domain of the secret (which we assume to know) and suppose we apply a $2$-step attack.
Here, we choose two public inputs such that we slice the domain such that we retrive the secret key depending on the cost measurement.
Given enough $k$, it is possible to retreive the secret key completely however it is not practical as search tree size grows exponentially.
However, if we simply choose $k$ independent public inputs then narrowing down to the secret is very hard.
On the other hand, choosing public inputs based on previous cost observations make it very easy.
For example, for the same example we discussed XXX

In symbolic analysis, we make both the secret and public inputs symbolic.
Usually symbolic execution is used for synthesis of best side channel attack (considering it is expensive and not so scalable), however here we will only investigate the scalability of \texttt{Klee}.
But it is important to note that in symbolic analysis, which \emph{encodes} all paths in terms of path conditions, it is possible to identify the leakage and thus synthesize best side channel attack.

XXX

\subsection{Software: \texttt{Klee} and our experimentation setup}
\label{subsec:softwares}

Docker usage:
\begin{enumerate}
    \item \texttt{sudo docker run --rm -ti --ulimit='stack=-1:-1' klee/klee}
    \item More information on docker available in \href{http://klee.github.io/releases/docs/v1.3.0/docker/}{github}
\end{enumerate}

Instructions to use for compilation, run and looking at testcases:
\begin{verbatim}
clang -I ../../include -emit-llvm -c -g ./examples/project_proto/symbolic_attack.c
klee -emit-all-errors symbolic_attack.bc 1 6 2
ktest-tool test000018.ktest
\end{verbatim}


\subsection{Data and analysis}
\label{subsec:dataandanalysis}

XXX

\newpage


\bibliographystyle{plain}
\bibliography{references}
\end{document}
